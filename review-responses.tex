\begin{minipage}[b]{2.5in}
    Resubmission Cover Letter \\
    {\it Molecular Biology and Evolution}
  \end{minipage}
  \hfill
  \begin{minipage}[b]{2.5in}
      Andrew Kern (on behalf of all authors)  \\
    \today
  \end{minipage}
  
  \vskip 2em
  
  \noindent
  {\bf To the Editor(s) -- }
  
  \vskip 1em
  
  We are writing to submit a revised version of our manuscript,
  MBE-25-0312
  titled
  ``Accessible, realistic genome simulation with selection using stdpopsim''.
  We appreciate the thoughtful comments by the reviewers.
  We've done our best to address all the comments,
  as detailed in our Response to Reviewers.
  
  In addition to a Response to Reviewers (in which page/line numbers refer to the revised manuscript file),
  we also provide a pdf with differences to the previously submitted version highlighted.
  
  
  \vskip 2em
  
  \noindent \hspace{4em}
  \begin{minipage}{3in}
  \noindent
  {\bf Sincerely,}
  
  \vskip 2em
  
  {\bf
     Andrew Kern (on behalf of all authors)
  }\\
  \end{minipage}
  
  \vskip 4em
  
  \pagebreak
  
  %%%%%%%%%%%%%%
  \reviewersection{AE}
  \begin{quote}
    \color{gray}
    Two experts in population genetics examined this manuscript and gave Very High and Medium marks in the Impact of the resource reported. 
    Despite this mixed evaluations, I think this extension of stdpopsim contributed by many authors should be very useful in the community of population genomics. 
    However, referees found many places in the manuscript that need clarification. 
    These comments should be addressed in revision. 
  \end{quote}
  
  Thanks for your comments on our paper.
  
  \begin{point}{\revref} % 
    In addition, regarding the point discussed in lines 368-373, a recent paper Marsh \& Johri 2024 MBE 41(7):msae118 
    might be acknowledged because it also developed simulations including the annonation for D. mel and humans and also mutation and recombination rate heterogeneity. 
  \end{point}
  
  \reply{
    Thanks for the suggestion. We have added this reference to the discussion section. \revref
  }
  
  \reviewersection{1}
  \begin{quote}
    \color{gray}
    In this manuscript, a large team of authors (Gower, Pope, Rodrigues, Tittes et al) present the next 
    set of developments for the stdpopsim software, which facilitates "standard" population-genetic 
    simulations that can be set to follow estimated demographic and genetic (mutation rates, recombination maps) 
    parameters in a number of species and populations, as well as a framework for entering new species and populations 
    into the collection. stdpopsim has become an important tool in population-genetic research. 
    The authors here represent a large community of people invested in the maintenance and improvement of the software, 
    which is open and undergoes thorough unit testing. 
    The addition of selection to stdpopsim's capabilities is a substantial advance warranting a new publication. 
    MBE also seems to me an appropriate outlet.  
\end{quote}
  
  \begin{quote}
    \color{gray}
    In addition to announcing the new features available in stdpopsim, the paper describes three applications of the software: 
    one on demographic inference in the presence of background selection, another on DFE estimation, and a third on detection of selective sweeps. 
    Two of these contrast the results obtained using human vs. vaquita porpoise parameters.  
\end{quote}
  
  \begin{quote}
    \color{gray}
    I am generally supportive of the paper and of the broader stdpopsim project. 
    That said, I do have some constructive thoughts and questions:  \end{quote}
  
  Thanks -- we're glad you appreciate both the project and the paper.
  
  \begin{point}{}
    Perhaps my main constructive reaction is that this paper is a bit light on information about the ways in which the new selection features are implemented. 
    There is a section "implementing selection in stdpopsim", but it is not always easy to figure out from the description here how things work under the hood. 
    Instead, the section is more like a short tour of the new features aimed at users. 
    Moreover, there is no section of the methods that really goes over the details of implementation. 
    The ms says that SLiM is the backend, but there aren't many details about how information passed into stdpopsim is translated into SLiM. 
    It's also not clear to me how the selective sweep mode is implemented, given that it gives users the option to condition on some range of starting and ending allele frequencies. 
    I assume that the approach is to run full SLiM iterations until the sweep ends up as desired (though it may not be)? 
    But one might also simulate the allele-frequency trajectory for the sweeping allele alone until one "works", 
    and then do coalescent simulations to fill in flanking variation conditional on the allele-frequency trajectory, 
    a la Hudson \& Kaplan '88 (though I don't quite see how one would do this simultaneously with background selection, 
    which the sweep simulations include). In any case, I'm not sure how the conditions are being achieved from reading the paper. 
    (I've refrained from looking at the online documentation for stdpopsim 0.3.0, where there may be more answers, in order to review the paper as an independent product.) 
    I don't think the paper necessarily needs loads more text on this, but I think there should be more information hinting at implementation in main text and probably a new methods subsection with further details.  
\end{point}
  
  \reply{
     %IG: I added this because the section I added for comment 1.5 below seems relevant here as well.
     %    I think that this addresses the first part of the rev's comment, but details need to also be added for selective sweeps
     We added a new section to the methods that provides detials on how DFE models are implemented using DFE objects \revref.
     The end of this section contains a brief description of how the simulation engine uses the information specified in a given DFE object when sampling a new mutation.
     TBD -- DETAILS ABOUT SELECTIVE SWEEPS.

  }
  
  
  \begin{point}{\revref} 
    Relatedly, why is the dominance coefficient currently limited to between 0 and 1? It's fine that it is for now, but I was curious what makes it more difficult to include other values of h.
  \end{point}
  
  \reply{
      This was a mistake in the original submission. The dominance coefficient is not limited to values between 0 and 1, and the 
      implementation allows for overdominance and underdominance. This clause has been removed \revref,
      %IG: I added a note about this in the 2nd part of the newly added methods section
      and in the newly added methods section about DFE models, we added an explicit note that dominance coefficients can be negative or larger than 1. 
  }
  
  \begin{point}{} 
    I think it would be helpful to say a bit more about the vaquita earlier in the manuscript. Although some of the authors have worked on the vaquita recently, it is still a fairly obscure organism, 
    so it would be worth describing its unfortunate situation and why it is an interesting comparison with humans throughout these simulations.  
\end{point}
  
  \reply{
     TODO. 
  }
  
  \begin{point}{} % 
    It seems as though everything on the backend is being done by SLiM in the demonstrations here 
    (unless I'm missing something about how the sweeps are done), so the runtimes are not really the product of the effort of the authors here. 
    Still, I think it would be good to provide a bit more information on the runtimes for prospective users to get a sense of what doing selection in stdpopsim entails. 
    (It's mentioned that the sweeps were time-consuming, but details aren't given as far as I can tell.)  
\end{point}
  
  \reply{
      TODO. 
  }
  
  \begin{point}{} % line 577
    Point of clarification about the flexibility of DFE specification: Is the DFE allowed to be a mixture distribution in a single genomic region? 
    Generally, the paper mentions users can provide a custom DFE, but what does that mean in practice? Does it have to be of a particular distribution family or form? 
    (i.e. the paper mentions "fixed value, normally-distributed, gamma-distributed"). 
    The vaquita DFE seems to be a mixture of several nonoverlapping DFEs with different h values, which suggests to me substantial flexibility, 
    but I'm not quite clear on the limits. Most of the figures feature fairly few replicates, which makes me think that these simulations are necessarily time/compute intensive.
  \end{point}
  
  \reply{
     We added a section to the methods with more details on \stdpopsim's DFE class, which answers these questions \revref.
     In particular, we clarified that each DFE model is a mixture of different mutation types,
     and we specified the distribution families that are currently supported by \stdpopsim.
     We also described in greater detail how the relationship between dominance and selection coefficients is implemented,
     which as the reviewer guessed, is done separately for each mutation type in the mixture.
     This framework was designed to be very flexible, while allowing the implementation of most DFE models with few lines of code.
  }

  \begin{point}{\revref} % line 577
    Sentence starting on line 383 about difficulty identifying h and s separately seems to be missing some important citations, 
    e.g. Fuller et al 2019 (pmid 30962618) and citations therein, including Simons et al 2014; Wright 1937.
  \end{point}
  
  \reply{
    TODO. \revref
  }

  \begin{point}{} % line 577
    It might bear further comment that the gamma distribution is not flexible enough to accommodate the distribution of hs in the vaquita simulation 
    (which is what the DFE inference methods can really see), even if the s values are gamma distributed (bc of the big jump caused in hs by the switch in dominance). 
    This is interesting given how often gammas are used in DFE inference. The paper does mention that using h=1/2 as an assumption can lead to bad estimation of s, 
    but the point that a relationship between s and h can cause hs to be non-gamma even when s is gamma is perhaps a little more subtle. 
    Generally, I thought the lessons learned from the DFE exercise here could be put more clearly.
  \end{point}
  
  \reply{
    Following this insightful comment, we added a few sentences to the discussion addressing this issue \revref:
    %IG: I added this quote of the text I added to the discussion to clarify our response to the reviewer.
    %    We'll need to update after we finalize the version in the revision.
    \textit{``
    Our analysis suggests that the distributions of $s$ estimated by these methods
    provide a reasonable approximation for the true distribution of $hs$ (Figure \ref{fig:vaquita-dfe}C).
    This points to a notable limitation of most existing DFE inference methods,
    which assume that selection coefficients are gamma-distributed.
    % DRS: see comment in main text on whether the limitation is the gamma for s or the fact that h is fixed
    As shown in Figure \ref{fig:vaquita-dfe}C,
    while a gamma distribution might fit the distribution of $s$, a relationship between $h$ and $s$ might result in
    $hs$ which is not well approximated by a gamma distribution.
    "}
  }
  
  \begin{point}{\revref} 
    Related to the previous point, I found it a bit odd to that both the style of the DFE (and in particular the h/s relationship) 
    and the demography switched when comparing vaquita and human DFE. 
    It seemed like a missed opportunity to do something hinted at earlier in the manuscript: 
    namely, to try pairing demographies and DFEs from different species. 
    I.e. it would be interesting in this context to look at simulations that have the human demography and vaquita DFE and vice versa, 
    even if only in a supplementary figure.
  \end{point}
  
  \reply{
    Thank you for the suggestion. 
    While it is possible to mix demographies and DFEs in stdpopsim catalog and study the effects, 
    this is outside of the motivation for this paper.
    Varying both DFE and demography in a systematic way and studying the effects would be interesting and useful, 
    but there is less to learn from only swapping the DFE and demography of two species in this context.
    We have clarified the point in the manuscript that when a given species of interest does not have published demographic and DFE inferences, 
    one can be borrowed from another species as a next best option.  
  }
  
  \begin{point}{\revref} % line 577
    Figure S1, right panel. Is pi being computed from all the genomes (CEU/CHB/YRI)? 
    I think it's okay given the point that's being made, 
    but it might strike some readers as odd given how much is often made of the pi differences 
    in these groups under the estimated demography, so maybe it should be said explicitly what's being done.
  \end{point}
  
  \reply{
    Thanks for the suggestion. In Fig. S1, $\pi$ is computed across all simulated genomes from all three populations,
    as we wish to focus on the comparison between diversity estimates in different genomic segments in different simulation scenarios.
    As the results are qualitatively similar when examining each of the three populations separately,
    we chose to summarize them in a single plot using a measures of $\pi$ across all three populations.
    We have updated the figure legend to clarify this \revref.
  }
  
  \begin{point}{}
    Last comment, and this may be an idiosyncratic opinion on my part, so I do not expect the authors to implement, 
    respond to, or even quote this point in the response letter. I'm merely offering a reaction. 
    In my opinion, three motifs are at a slightly awkward level of detail/coverage. 
    On one hand, they're not enough to please partisans or enthusiasts of any of these three areas. 
    (The paper is clear about the fact that they're not meant to be thorough benchmarks.) 
    On the other hand, they are a big part of the paper, substantial on their own, 
    and the bulk of the discussion is given over to their substantive results rather than to stdpopsim per se. 
    I personally find many of the results interesting, but I also wondered whether the paper might be stronger 
    if these segments were streamlined. One way to do this might be to focus less on the three topics themselves, 
    but to change the emphasis to how selection results change when the species changes, 
    i.e. to focus the thread of the results primarily on a quick inspection of the human/vaquita differences
    (one might imagine adding in Drosophila too in such a framing) rather than on separate questions about the effect of 
    BGS on demographic estimation and sweep inference, and on the effect of h-s relationship on DFE inference. 
    Again, this is merely offered in the spirit of a friendly comment, and I expect no response.
  \end{point}
  
  \reply{
    %IG: I don't think we need to respond to this, but I did try to draft a polite response that clarifies how we see the paper
    %    Feel free to shorten/remove this, if you think this is calling for trouble
    We appreciate this comment.
    The structure of this manuscript is indeed something we put a lot of thought into because of its somewhat unorthodox nature.
    The main objective of the manuscript is to report on new capabilities of \stdpopsim,
    so that the broad scientific community could use them to produce novel scientific discoveries.
    The analyses presented in the manuscript are meant to demonstrate how features added to \stdpopsim
    could be used to evaluate inference methods and test hypotheses regarding the effect of natural selection on patterns of diversity.
    Consequently, the results of these analyses largely lead to more open questions than resolved questions,
    and this is reflected in the discussion section, as noted by the reviewer.
    While this is a bit unorthodox, we found this type of structure to be very useful in highlighting
    \stdpopsim's features in our two previous publications \citep{adrion2020community,lauterbur2023expanding}.
    We thus structured this manuscript similarly.
    
  }
  

  
  %%%%%%%%%%%%%%
  \reviewersection{2}
  \begin{quote}
    \color{gray}
    In this manuscript, Gower et al. extend the functionality of stdpopsim to include selection. 
    They test the performance of a few different methods that infer historical population sizes, 
    the distribution of fitness effects of new deleterious mutations, and those that detect selective sweeps using the simulations in SLiM. 
    The added functionality will be useful to the users of stdpopsim.
\end{quote}
  
  Thanks for your constructive comments on our paper.
  
  \begin{point}{\revref} % \&75
    How do you deal with overlapping annotations in a gff?
  \end{point}
  
  \reply{
    TODO. \revref
  }
  
  \begin{point}{}
    The inference of the deleterious DFE with GRAPES is quite inaccurate. 
    Is it possible that there is a factor of 2 error involved?
  \end{point}
  
  \reply{
As noted in the paper, conversion of selection inferences between tools is challenging.
We thus consulted with the developers of \polydfe and \grapes to confirm our conversions.
(The developer of \dadicli is an author on this paper.)
The inaccuracy of \grapes is a surprise to us.
Note that \grapes is also less accurate than \polydfe and \dadicli in inferring the shape of the distribution (Fig.~\ref{fig:homsap-dfe.ooa}B), which suggests the inaccuracy is not simply a scaling issue.
  }
  
  \begin{point}{\revref} % 21:
    Figure 3c does not have any error bars, it’s difficult to conclude if the inference is actually significantly different from the truth. 
    Which “N” was used to obtain estimates of “s” in Figure 3a? 
    This should be mentioned in the legend.  \end{point}
  
  \reply{
  %IG: I don't think that error bars are appropriate here and I tried to explain why
  The main purpose of the plots in Figure 3C is to show the actual differences between simulated
  and inferred distributions of $s$, since differences in mean and shape (shown in panels A-B) are not
  always very informative or insightful.
  To do this, we considered the average inferred DFE across the three replicate simualtions.
  We do not see much use in adding error bars to these plots reflecting the variation across the three replicates.
  Because of the somewhat noisy nature of this comparison, we only draw conclusions regarding qualitative differences
  between inferred and simulated distributions, namely,
  that  \polydfe and \dadicli tend to over-estimate the fraction of mutations under strong selection
  amd \grapes tends to over-estimate the fraction of mutations under weak selection.
  %IG: please check that this is correct
  Regarding the estimates of $\text{E}[s]$ shown in Figure 3A (and also Figure 4A and associated supplementary figures),
  these are typically values produced directly by the inference methods.
  In the last paragraph of the methods section on DFE inference, we provide a complete description of the scaling
  we apply to the output of each method so that they are directly comparable to $s$ as simulated in \slim.
  }
  
  % INTRO
  
  \begin{point}{\revref} % 24:
    You cite only Gillespie 1991 on line 56. 
    Requires many more citations, with at least one to Kimura.  
\end{point}

  \reply{
  We have elaborated the list of citations for this point and have now cited Kimura as well. \revref
  }
  
  \begin{point}{} % 27:
    What fraction of the genome is under selection in the human genome you simulated?
  \end{point}
  
  \reply{
 For simulations with selection we assumed that selected mutations could only fall within exonic regions,
 thus approximately 2.1\% of the genome is under selection in human simulations and 2.2\% in vaquita simulations.
  }
  
  \begin{point}{\revref} % 47:
    This is somewhat surprising given that background selection is expected to somewhat interfere with the sweep signature.” (line 349-350) $\to$ I'm not sure what this means. 
    It's unclear to me why background selection will interfere with a sweep signature. 
    Do you have a citation for this?  
\end{point}
  
  \reply{
  We have added a citation to a review paper from Wolfgang Stephan that discusses this point from a number of different perspectives \revref.
  }
  
